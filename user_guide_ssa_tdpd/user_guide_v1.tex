\documentclass[a4paper,12pt,oneside]{report}
\parindent 0 pt
\parskip 12 pt
\baselineskip 0pt
\usepackage{graphicx}
\usepackage{tabularx}
\usepackage{amsmath}
\usepackage{fancyhdr}
\usepackage[hidelinks]{hyperref}
\usepackage{listings}
\usepackage{color}
\usepackage{authblk}

\definecolor{dkgreen}{rgb}{0,0.6,0}
\definecolor{gray}{rgb}{0.5,0.5,0.5}
\definecolor{mauve}{rgb}{0.58,0,0.82}
\lstset{frame=tb,
  language=bash,
  aboveskip=3mm,
  belowskip=3mm,
  showstringspaces=false,
  columns=flexible,
  basicstyle={\small\ttfamily},
  numbers=none,
  numberstyle=\tiny\color{gray},
  keywordstyle=\color{black},
  commentstyle=\color{dkgreen},
  stringstyle=\color{mauve},
  breaklines=true,
  breakatwhitespace=true,
  tabsize=3
}

\pagestyle{fancy}
\fancyhf{}
\fancyhead[EL]{\nouppercase\leftmark}
\fancyhead[OR]{\nouppercase\rightmark}
\fancyhead[ER,OL]{\thepage}

\renewcommand{\thesection}{\arabic{section}}

\makeatletter
\newcommand{\chapterauthor}[1]{%
  {\parindent0pt\vspace*{-5pt}%
  \linespread{1.1}\scshape#1%
  \par\nobreak\vspace*{25pt}}
  \@afterheading%
}
\makeatother

\begin{document}


\chapter*{User Guide -- SSA-tDPD module (v1.0)}

%\chapterauthor{by Brian Drawert, Bruno Jacob, Zhen Li}

\section{Installation process}


\subsection{Installing Git}
\begin{itemize}
\item In a terminal, type: \\

 \texttt{sudo apt-get install git}\\
 
This will install git in your system. You have to create an account on Bitbucket (\url{https://bitbucket.org/}) to be able to access the ssa\_tDPD repository.


\item Now that Git is installed, at the first time you run any git command, the system will ask you a few things, so that the commit messages are correctly configured. Type:\\

 \texttt{git config --global user.name "Your Name"}\\
 
 \texttt{git config --global user.email "youremail@domain.com"}\\


\end{itemize}


\pagebreak




\subsection{Installing VTK}
\begin{itemize}
\item In a terminal, type: \\

 \texttt{sudo apt-get install vtk}\\
 
This will install the VTK library in your system.

\end{itemize}


\subsection{Installing Paraview}
\begin{itemize}
\item In a terminal, type:\\

 \texttt{sudo apt-get install paraview}\\

This will install Paraview in your system. Paraview is a user-friendly VTK reader, used for post-processing Lammps results.


\end{itemize}



\subsection{Optional: Installing MPI}
\begin{itemize}
\item In a terminal, type: \\

  \texttt{sudo apt-get install libopenmpi-dev openmpi-bin}\\
  
This will install OpenMPI in your system. It allows running Lammps using more than one processor.

\end{itemize}


\pagebreak

\subsection{Installing Lammps}
\begin{itemize}
\item In a terminal, type:\\

\texttt{git clone git@bitbucket.org:bdrawert/ssa\_tdpd.git}\\
  
to clone the repository from Bitbucket.


\item Open the cloned folder, and navigate to the ./lib/vtk directory \\

\texttt{cd ssa\_tdpd/lib/vtk}\\

\item Edit the file Makefile.lammps accordingly:\\


\texttt{vtk\_SYSINC = -I/usr/include/vtk-X.X\\
vtk\_SYSLIB = -lvtkCommonCore -lvtkIOCore -lvtkCommonDataModel -lvtkIOXML -lvtkIOLegacy -lvtkIOParallelXML\\
vtk\_SYSPATH = -L/usr/lib64/vtk}

where X.X is the version of the VTK library installed in your system.

\end{itemize}


\section{Compiling Lammps}
\begin{itemize}

\item Open the ssa\_tdpd folder, and navigate to the ./src directory. Then type: \\

  \texttt{make yes-USER-VTK}\\

This will install the VTK package in Lammmps.


\item Still in the /src directory, type:\\

  \texttt{make mpi}\\
  
if MPI is installed. Alternatively, type \texttt{make serial} to install the serial version.\\

  
\item If no error is displayed, the C++ compiler will generate a binary named \texttt{lmp\_mpi} (or \texttt{lmp\_serial}, if the case). 
  
\end{itemize}



\section{Running cases}

\begin{itemize}

\item To run cases in parallel (MPI), use the following syntax:\\
  
  \texttt{mpirun -np X path\_to\_lmp\_mpi -in path\_to\_input\_file}\\
  
where \texttt{X} is the number of MPI processes desired (for optimal performance, this number must match the number of physical processors in your machine).
Note that \texttt{path\_to\_lmp\_mpi} must be replaced by the path where the \texttt{lmp\_mpi} file is located, and \texttt{path\_to\_input\_file} must be replaced by the path of the input file.

\item Alternatively, to run cases in serial, use the following syntax:\\
  
  \texttt{./path\_to\_lmp\_mpi -in path\_to\_input\_file}\\
  
  
\end{itemize}

\pagebreak

\section{Input file structure}

  Example: flow past sphere
  
\begin{lstlisting}

# DPD flow past a sphere (with diffusion of one species)

####################################
# Lammps setup
####################################
dimension  2 #enforces 2D simulation
units  lj #reduced Lennard-Jones units
comm_modify mode single vel yes
atom_style  ssa_tdpd/atomic #type of atom
neighbor  0.3 bin #creates neighbor list
neigh_modify  delay 0 every 1 check yes


####################################
# Temporal integration setup
####################################
variable  dt equal 0.0001 #time step
variable  nt equal 15000 #number of time steps
variable  freq_lagrangian equal 100 #freq. writing results (file)
variable  freq_screen equal 100 #freq. writing results (screen)

####################################
# Domain setup
####################################
boundary  p f p #(x,y,z); p = periodic, f = fixed
variable  xmin equal -200 # minimum value of x
variable  xmax equal 200 # maximum value of x
variable  ymin equal -50 # minimum value of y
variable  ymax equal 50 # maximum value of y
variable  zmin equal -0.01 # minimum value of z
variable  zmax equal 0.01 # maximum value of z
variable  radius_external equal 2.2 # external radius
variable  radius_internal equal 2.0 # internal radius
region  domain1 block ${xmin} ${xmax}  ${ymin}  ${ymax} ${zmin} ${zmax}  units box
region	sphere_region sphere 0 0 0 ${radius_external} units box
region  domain union 2 domain1 sphere_region
create_box 2 domain  #(Number of types ; region) 

####################################
# Creates atoms and regions
####################################
lattice  sq 1 #(lattice topology; number density)
create_atoms  1 region domain1 #(type; region region-ID)
create_atoms  2 random 2000 58131124 sphere_region #(type; random # particles; seed; region-ID)
mass  * 0.01 #mass of particles
newton  on
pair_style  ssa_tdpd 1.58 354655456 #DPD parameters (rc, seed)
#	    i  j  a     ga   si   s1    rc    rcc   kC  k0     s2 
pair_coeff  *  *  1000   4.5  3.0  0.41  1.58  1.58  0  100.0  2.0

####################################
# Defines sphere region
####################################
# Removes a spherical region
group  sphere region sphere_region #defines a group of atoms
region  void1 sphere 0 0 0 ${radius_internal} units box #creates region
delete_atoms  region void1 #deletes atoms within a region
group  flow subtract all sphere #defines a group of atoms
velocity  sphere set 0.0 0.0 0.0 
fix  spherewall flow indent 10 sphere 0 0 0 ${radius_external} units box
set	 group sphere type 2

####################################
# Initial velocity and concentration fields
####################################
# Velocity
variable U0 equal 208.0  #(label, initial velocity)
velocity all set ${U0} 0.0 0.0 #(group-ID, set vx vy vz)
# Concentration
set	group sphere ssa_tdpd/C 0.0 #(group group-ID, style, value)

####################################
# Info on screen
####################################
thermo          ${freq_screen} 
thermo_style    custom step temp press pe ke

####################################
# Integration of particles' position, velocity, concentration
####################################
fix  pos_vel flow nve  #(label, group-ID,style)
fix  conc all ssa_tdpd_verlet #(label, group-ID,style)

####################################
# Forcing zone
####################################
fix  forcing1 all ssa_tdpd/forcing 1 0 rectangle -195 25  5.0 25 1.0 #(label, group-ID, style, frequency, species rank, geometry, centerX, centerY, length, width, value)
fix  forcing2 all ssa_tdpd/forcing 1 0 rectangle -195 -25 5.0 25 0.0 #(label, group-ID, style, frequency, species rank, geometry, centerX, centerY, length, width, value)

####################################
# Enforce boundary conditions
####################################
fix  BCy all ssa_tdpd/reflect 1 ${ymin} ${ymax} #(label, group-ID, style, species, rank coordinate (x=0, y=1, z=2), coord_min, coord_max)

####################################
# Output results
####################################
dump  dmpvtk all ssa_tdpd/vtk ${freq_lagrangian} dump*.vtk id type vx vy vz C1 #(label, group-ID, style, frequency, filenames, variables to print)

####################################
# Run simulation
####################################
timestep  ${dt}
run  ${nt}

\end{lstlisting}
 
 
 


\end{document}

